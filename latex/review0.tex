\documentclass{beamer}
\usepackage[utf8]{inputenc}
\usepackage{amsmath}
\usepackage{wasysym}
\usepackage{listings}

\usetheme{Madrid}
\usecolortheme{default}

\lstset{
  basicstyle=\small,
}

\title[CSC 212 Review 1]
{Mathematical Analysis of Algorithms}

\author
{Calvin Higgins}

\institute[]
{
  Department of Computer Science and Statistics\\
  University of Rhode Island
}

\AtBeginSection[]
{
  \begin{frame}
    \frametitle{Table of Contents}
    \tableofcontents[currentsection]
  \end{frame}
}

\begin{document}

% ------------------------------------------------------------
% Slide
% ------------------------------------------------------------

\frame{\titlepage}

% ------------------------------------------------------------
% Slide
% ------------------------------------------------------------

\begin{frame}
\frametitle{The Big Picture}

    \begin{center}
        \Huge
        \textcolor{red}{\textbf{What is algorithm analysis?}}
    \end{center}

    \vspace{1em}

    \begin{center}
        \Huge
        \textcolor{red}{\textbf{Why do we analyze algorithms?}}
    \end{center}

\end{frame}


% ------------------------------------------------------------
% Slide
% ------------------------------------------------------------

\begin{frame}
\frametitle{The Big Picture}
    \begin{center}
        \Large
        \textcolor{blue}{\textbf{Algorithm analysis is the prediction and comparison of algorithm performance.}}
    \end{center}

    \vspace{1em}

    \begin{center}
        \Large
        \textcolor{blue}{\textbf{Algorithm analysis lets us choose or design the best (or good enough) algorithm for a given problem.}}
    \end{center}
    
\end{frame}

% ------------------------------------------------------------
% Slide
% ------------------------------------------------------------

\begin{frame}
\frametitle{Mathematical Algorithm Analysis}

    \begin{center}
        \Huge
        \textcolor{red}{\textbf{How do we mathematically analyze algorithms?}}
    \end{center}

    \vspace{1em}

    \begin{center}
        \Huge
        \textcolor{red}{\textbf{Give a step-by-step procedure.}}
    \end{center}
    
\end{frame}

% ------------------------------------------------------------
% Slide
% ------------------------------------------------------------

\begin{frame}
\frametitle{Mathematical Algorithm Analysis}
    \textcolor{blue}{\textbf{How to Analyze an Algorithm:}}
    \begin{enumerate}
        \item Define a reasonable \textbf{model of computation} (\textbf{cost model}).
        \begin{enumerate}
            \item What are the \textbf{basic operations}? 
            \item How much does each basic operation cost?
        \end{enumerate}
        
        % TODO: case analysis

        \item Model the algorithm's cost with a function $T(n)$.
        \begin{enumerate}
            \item How many basic operations are performed for an input of size $n$?
        \end{enumerate}

        \item Simplify $T(n)$.

        \item Classify $T(n)$'s growth rate.
        \begin{enumerate}
            \item How quickly does $T(n)$ grow?
        \end{enumerate}

        \item Interpret $T(n)$'s growth rate.
        \begin{enumerate}
            \item How suitable is the algorithm for my problem?
        \end{enumerate}
    \end{enumerate}
\end{frame}

% ------------------------------------------------------------
% Slide
% ------------------------------------------------------------

\begin{frame}[fragile]
\frametitle{Defining a Model of Computation}

        \begin{lstlisting}[language=C++]
int foo(int n) {
    int A = new int[n];
    
    A[0] = 1;
    for (int i = 1; i < n; i++)
        A[i] = A[i - 1] * (i + 1);

    int sum = 0;
    for (int i = 0; i < n; i++)
        for (int j = 0; j < i; j++)
            sum += A[j];

    delete[] A;
    return sum;
}
        \end{lstlisting}

    \begin{center}
        \Large
        \textcolor{red}{\textbf{List as many basic operations as you can think of!}}
    \end{center}

\end{frame}

% ------------------------------------------------------------
% Slide
% ------------------------------------------------------------

\begin{frame}[fragile]
\frametitle{Defining a Model of Computation}
    \begin{columns}[T] 
    
        \begin{column}{0.65\textwidth}
            \begin{lstlisting}[language=C++]
int foo(int n) {
    int* A = new int[n];
    
    A[0] = 1;
    for (int i = 1; i < n; i++)
        A[i] = A[i - 1] * (i + 1);

    int sum = 0;
    for (int i = 0; i < n; i++)
        for (int j = 0; j < i; j++)
            sum += A[j];

    delete[] A;
    return sum;
}
           \end{lstlisting}
        \end{column}
    
        \begin{column}{0.35\textwidth}
            \textcolor{blue}{\textbf{Basic Operations:}}
            \begin{enumerate}
                \item \textcolor{blue}{Additions}
                \item \textcolor{blue}{Multiplications}
                \item \textcolor{blue}{Comparisons}
                \item \textcolor{blue}{Branches}
                \item \textcolor{blue}{Local variables}
                \item \textcolor{blue}{Memory allocations}
                \item \textcolor{blue}{Allocated memory}
                \item \textcolor{blue}{Loads}
                \item \textcolor{blue}{Stores}
                \item \textcolor{blue}{Assignments}
                \item \textcolor{blue}{...}
            \end{enumerate}
        \end{column}
    \end{columns}

    \begin{center}
        \Large
        \textcolor{red}{\textbf{Which basic operations are most reasonable? Why?}}
    \end{center}
\end{frame}

% ------------------------------------------------------------
% Slide
% ------------------------------------------------------------

\begin{frame}[fragile]
\frametitle{Modeling the Algorithm's Cost}    
    \begin{lstlisting}[language=C++, mathescape=true]
int foo(int n) {
    int* A = new int[n];
    
    A[0] = 1;
    for (int i = 1; i < n; i++)     // $\textcolor{red}{\text{Additions?}}$
        A[i] = A[i - 1] * (i + 1);  // $\textcolor{red}{\text{Additions?}}$

    int sum = 0;
    for (int i = 0; i < n; i++)     // $\textcolor{red}{\text{Additions?}}$
        for (int j = 0; j < i; j++) // $\textcolor{red}{\text{Additions?}}$
            sum += A[j];            // $\textcolor{red}{\text{Additions?}}$  

    delete[] A;
    return sum;
}
    \end{lstlisting}

    \begin{center}
        \Large
        \textcolor{red}{\textbf{How many additions?} $T(n) = \; ?$}
    \end{center}
\end{frame}

% ------------------------------------------------------------
% Slide
% ------------------------------------------------------------

\begin{frame}[fragile]
\frametitle{Modeling the Algorithm's Cost}    
    \begin{lstlisting}[language=C++, mathescape=true]
int foo(int n) {
    int* A = new int[n];
    
    A[0] = 1;
    for (int i = 1; i < n; i++)     // $\textcolor{blue}{\sum_{i=1}^{n - 1} 1 \text{ additions}}$
        A[i] = A[i - 1] * (i + 1);  // $\textcolor{blue}{\sum_{i=1}^{n - 1} 2 \text{ additions}}$

    int sum = 0;
    for (int i = 0; i < n; i++)     // $\textcolor{blue}{\sum_{i=0}^{n - 1} 1 \text{ additions}}$
        for (int j = 0; j < i; j++) // $\textcolor{blue}{\sum_{i=0}^{n - 1} \sum_{j=0}^{i} 1 \text{ additions}}$
            sum += A[j];            // $\textcolor{blue}{\sum_{i=0}^{n - 1} \sum_{j=0}^{i - 1} 1 \text{ additions}}$  

    delete[] A;
    return sum;
}
    \end{lstlisting}

    \begin{center}
        \textcolor{blue}{$T(n) = \sum_{i=1}^{n-1} 1 + \sum_{i=1}^{n - 1} 2 + \sum_{i=0}^{n-1} 1 + \sum_{i=0}^{n - 1} \sum_{j=0}^{i} 1 + \sum_{i=0}^{n - 1} \sum_{j=0}^{i - 1} 1$}
    \end{center}
\end{frame}

% ------------------------------------------------------------
% Slide
% ------------------------------------------------------------

\begin{frame}[fragile]
\frametitle{Modeling the Algorithm's Cost}
    \begin{lstlisting}[language=C++, mathescape=true]
int foo(int n) {
    int* A = new int[n];
    
    A[0] = 1;
    for (int i = 1; i < n; i++)    
        A[i] = A[i - 1] * (i + 1); // $\textcolor{red}{\text{Multiplications?}}$

    int sum = 0;
    for (int i = 0; i < n; i++)
        for (int j = 0; j < i; j++)
            sum += A[j];              

    delete[] A;
    return sum;
}
   \end{lstlisting}

    \begin{center}
        \Large
        \textcolor{red}{\textbf{How many multiplications?} $ T(n) =\;?$}
    \end{center}
\end{frame}

% ------------------------------------------------------------
% Slide
% ------------------------------------------------------------

\begin{frame}[fragile]
\frametitle{Modeling the Algorithm's Cost}
    \begin{lstlisting}[language=C++, mathescape=true]
int foo(int n) {
    int* A = new int[n];
    
    A[0] = 1;
    for (int i = 1; i < n; i++)    
        A[i] = A[i - 1] * (i + 1); // $\textcolor{blue}{\sum_{i=1}^{n - 1} 1 \text{ multiplications}}$

    int sum = 0;
    for (int i = 0; i < n; i++)
        for (int j = 0; j < i; j++)
            sum += A[j];

    delete[] A;
    return sum;
}
   \end{lstlisting}

    \begin{center}
        \Large
        \textcolor{blue}{\textbf{Multiplications:} $ T(n) = \sum_{i=1}^{n - 1} 1$}
    \end{center}
\end{frame}


% ------------------------------------------------------------
% Slide
% ------------------------------------------------------------

\begin{frame}[fragile]
\frametitle{Modeling the Algorithm's Cost}
    \begin{lstlisting}[language=C++, mathescape=true]
int foo(int n) {
    int* A = new int[n];
    
    A[0] = 1;
    for (int i = 1; i < n; i++)     // $\textcolor{red}{\text{Comparisons?}}$
        A[i] = A[i - 1] * (i + 1);

    int sum = 0;
    for (int i = 0; i < n; i++)     // $\textcolor{red}{\text{Comparisons?}}$
        for (int j = 0; j < i; j++) // $\textcolor{red}{\text{Comparisons?}}$
            sum += A[j];            

    delete[] A;
    return sum;
}
   \end{lstlisting}

    \begin{center}
        \Large
        \textcolor{red}{\textbf{How many comparisons?} $ T(n) =\;?$}
    \end{center}
\end{frame}

% ------------------------------------------------------------
% Slide
% ------------------------------------------------------------

\begin{frame}[fragile]
\frametitle{Modeling the Algorithm's Cost}
    \begin{lstlisting}[language=C++, mathescape=true]
int foo(int n) {
    int* A = new int[n];
    
    A[0] = 1;
    for (int i = 1; i < n; i++)     // $\textcolor{blue}{\sum_{i=1}^{n} 1 \text{ comparisons}}$
        A[i] = A[i - 1] * (i + 1);

    int sum = 0;
    for (int i = 0; i < n; i++)     // $\textcolor{blue}{\sum_{i=0}^{n} 1 \text{ comparisons}}$
        for (int j = 0; j < i; j++) // $\textcolor{blue}{\sum_{i=0}^{n-1} \sum_{j=0}^{i} 1 \text{ comparisons}}$
            sum += A[j];    

    delete[] A;
    return sum;
}
    \end{lstlisting}

    \begin{center}
        \Large
        \textcolor{blue}{\textbf{Comparisons:} $ T(n) = \sum_{i=1}^{n} 1 + \sum_{i=0}^{n} 1 + \sum_{i=0}^{n-1} \sum_{j=0}^{i} 1$}
    \end{center}
\end{frame}

% ------------------------------------------------------------
% Slide
% ------------------------------------------------------------

\begin{frame}[fragile]
\frametitle{Modeling the Algorithm's Cost}
    \begin{lstlisting}[language=C++, mathescape=true]
int foo(int n) {
    int* A = new int[n];
    
    A[0] = 1;                       // $\textcolor{red}{\text{Memory accesses?}}$
    for (int i = 1; i < n; i++)    
        A[i] = A[i - 1] * (i + 1);  // $\textcolor{red}{\text{Memory accesses?}}$

    int sum = 0;
    for (int i = 0; i < n; i++)
        for (int j = 0; j < i; j++) 
            sum += A[j];            // $\textcolor{red}{\text{Memory accesses?}}$    

    delete[] A;
    return sum;
}
    \end{lstlisting}

    \begin{center}
        \Large
        \textcolor{red}{\textbf{How many memory accesses (indexing)?} $ T(n) =\;?$}
    \end{center}
\end{frame}

% ------------------------------------------------------------
% Slide
% ------------------------------------------------------------

\begin{frame}[fragile]
\frametitle{Modeling the Algorithm's Cost}
    \begin{lstlisting}[language=C++, mathescape=true]
int foo(int n) {
    int* A = new int[n];
    
    A[0] = 1;                       // $\textcolor{blue}{1 \text{ memory accesses}}$
    for (int i = 1; i < n; i++)    
        A[i] = A[i - 1] * (i + 1);  // $\textcolor{blue}{\sum_{i=1}^{n-1} 2 \text{ memory accesses}}$

    int sum = 0;
    for (int i = 0; i < n; i++)
        for (int j = 0; j < i; j++) 
            sum += A[j];            // $\textcolor{blue}{\sum_{i=0}^{n-1} \sum_{j=0}^{i - 1} 1 \text{ memory accesses}}$ 

    delete[] A;
    return sum;
}
    \end{lstlisting}

    \begin{center}
        \Large
        \textcolor{blue}{\textbf{Memory Accesses:} $T(n) = 1 + \sum_{i=1}^{n-1} 2 + \sum_{i=0}^{n-1} \sum_{j=0}^{i - 1} 1$}
    \end{center}
\end{frame}

% ------------------------------------------------------------
% Slide
% ------------------------------------------------------------

\begin{frame}[fragile]
\frametitle{Modeling the Algorithm's Cost}
    \begin{lstlisting}[language=C++, mathescape=true]
int foo(int n) {
    int* A = new int[n];            // $\textcolor{red}{\text{Assignments?}}$
    
    A[0] = 1;                       // $\textcolor{red}{\text{Assignments?}}$
    for (int i = 1; i < n; i++)     // $\textcolor{red}{\text{Assignments?}}$
        A[i] = A[i - 1] * (i + 1);  // $\textcolor{red}{\text{Assignments?}}$

    int sum = 0;                    // $\textcolor{red}{\text{Assignments?}}$
    for (int i = 0; i < n; i++)     // $\textcolor{red}{\text{Assignments?}}$
        for (int j = 0; j < i; j++) // $\textcolor{red}{\text{Assignments?}}$
            sum += A[j];            // $\textcolor{red}{\text{Assignments?}}$ 

    delete[] A;
    return sum;
}
    \end{lstlisting}

    \begin{center}
        \Large
        \textcolor{red}{\textbf{How many assignments?} $ T(n) =\;?$}
    \end{center}
\end{frame}

% ------------------------------------------------------------
% Slide
% ------------------------------------------------------------

\begin{frame}[fragile]
\frametitle{Modeling the Algorithm's Cost}
    \begin{lstlisting}[language=C++, mathescape=true]
int foo(int n) {
    int* A = new int[n];            // $\textcolor{blue}{1 \text{ assignments}}$
    
    A[0] = 1;                       // $\textcolor{blue}{1 \text{ assignments}}$
    for (int i = 1; i < n; i++)     // $\textcolor{blue}{\sum_{i=1}^{n-1} 1 \text{ assignments}}$
        A[i] = A[i - 1] * (i + 1);  // $\textcolor{blue}{\sum_{i=1}^{n-1} 1 \text{ assignments}}$

    int sum = 0;                    // $\textcolor{blue}{1 \text{ assignments}}$
    for (int i = 0; i < n; i++)     // $\textcolor{blue}{\sum_{i=0}^{n-1} 1 \text{ assignments}}$
        for (int j = 0; j < i; j++) // $\textcolor{blue}{\sum_{i=0}^{n-1} \sum_{j=0}^{i} 1 \text{ assignments}}$ 
            sum += A[j];            // $\textcolor{blue}{\sum_{i=0}^{n-1} \sum_{j=0}^{i - 1} 1 \text{ assignments}}$ 

    delete[] A;
    return sum;
}
   \end{lstlisting}

    \begin{center}
        \textcolor{blue}{$T(n) = 3 + \sum_{i=1}^{n} 1 + \sum_{i=1}^{n-1} 1 + \sum_{i=0}^{n} 1 + \sum_{i=0}^{n-1} \sum_{j=0}^{i} 1 + \sum_{i=0}^{n-1} \sum_{j=0}^{i - 1} 1$}
    \end{center}
\end{frame}

% ------------------------------------------------------------
% Slide
% ------------------------------------------------------------

\begin{frame}
\frametitle{Simplifying the Cost Function}
    \begin{block}{Addition Cost Function}
        $T(n) = \textcolor{red}{\sum\limits_{i=1}^{n-1} 1} + \sum\limits_{i=1}^{n - 1} 2 + \sum\limits_{i=0}^{n-1} 1 + \sum\limits_{i=0}^{n - 1} \sum\limits_{j=0}^{i} 1 + \sum\limits_{i=0}^{n - 1} \sum\limits_{j=0}^{i - 1} 1$
    \end{block}

    \begin{columns}[T] 
        \begin{column}{0.4\textwidth}
            \textbf{Simplification:}
            
            Apply \textcolor{blue}{(1)} with $m = n - 1$:
            \begin{align*}
                \textcolor{red}{\sum\limits_{i=1}^{n - 1} 1 = n-1}
            \end{align*}
        \end{column}
    
        \begin{column}{0.6\textwidth}
            \textbf{Identities:}
            \begin{enumerate}
                \item \textcolor{blue}{$\sum\limits_{i=1}^m 1 = m$}
                \item \textcolor{gray}{$\sum\limits_{i=1}^m i = \frac{m(m+1)}{2}$}
                \item \textcolor{gray}{$\sum\limits_{i=a}^b c f(i) = c \sum\limits_{i=a}^b f(i)$ where $c$ is a \textbf{constant} and $f$ is a function}
                \item \textcolor{gray}{$\sum\limits_{i=a}^b (f(i) + g(i)) = \sum\limits_{i=a}^b f(i) + \sum\limits_{i=a}^b g(i)$ where $f$ and $g$ are functions}
            \end{enumerate}
        \end{column}
    \end{columns}
\end{frame}

% ------------------------------------------------------------
% Slide
% ------------------------------------------------------------

\begin{frame}
\frametitle{Simplifying the Cost Function}
    \begin{block}{Addition Cost Function}
        $T(n) = (n - 1) + \textcolor{red}{\sum\limits_{i=1}^{n - 1} 2} + \sum\limits_{i=0}^{n-1} 1 + \sum\limits_{i=0}^{n - 1} \sum\limits_{j=0}^{i} 1 + \sum\limits_{i=0}^{n - 1} \sum\limits_{j=0}^{i - 1} 1$
    \end{block}

    \begin{columns}[T] 
        \begin{column}{0.4\textwidth}
            \textbf{Simplification:}
            \begin{align*}
                \textcolor{red}{\sum\limits_{i=1}^{n - 1} 2 = \; ?}
            \end{align*}
            \begin{center}
                \Large
                \textcolor{red}{\textbf{What identity should we use?}}
            \end{center}
        \end{column}
    
        \begin{column}{0.6\textwidth}
            \textbf{Identities:}
            \begin{enumerate}
                \item \textcolor{black}{$\sum\limits_{i=1}^m 1 = m$}
                \item \textcolor{black}{$\sum\limits_{i=1}^m i = \frac{m(m+1)}{2}$}
                \item \textcolor{black}{$\sum\limits_{i=a}^b c f(i) = c \sum\limits_{i=a}^b f(i)$ where $c$ is a \textbf{constant} and $f$ is a function}
                \item \textcolor{black}{$\sum\limits_{i=a}^b (f(i) + g(i)) = \sum\limits_{i=a}^b f(i) + \sum\limits_{i=a}^b g(i)$ where $f$ and $g$ are functions}
            \end{enumerate}
        \end{column}
    \end{columns}
\end{frame}

% ------------------------------------------------------------
% Slide
% ------------------------------------------------------------

\begin{frame}
\frametitle{Simplifying the Cost Function}
    \begin{block}{Addition Cost Function}
        $T(n) = (n-1) + \textcolor{red}{\sum\limits_{i=1}^{n - 1} 2} + \sum\limits_{i=0}^{n-1} 1 + \sum\limits_{i=0}^{n - 1} \sum\limits_{j=0}^{i} 1 + \sum\limits_{i=0}^{n - 1} \sum\limits_{j=0}^{i - 1} 1$
    \end{block}

    \begin{columns}[T] 
        \begin{column}{0.4\textwidth}
            \textbf{Simplification:}

            Apply \textcolor{blue}{(3)} with $a = 1$, $b = n - 1$, $c = 2$ and $f(i) = 1$
            \begin{align*}
                \textcolor{red}{\sum\limits_{i=1}^{n - 1} 2 = \sum\limits_{i=1}^{n - 1} 2 \cdot 1 = 2\sum\limits_{i=1}^{n - 1} 1}
            \end{align*}
            \begin{center}
                \Large
                \textcolor{red}{\textbf{What identity should we use?}}
            \end{center}
        \end{column}
    
        \begin{column}{0.6\textwidth}
            \textbf{Identities:}
            \begin{enumerate}
                \item \textcolor{gray}{$\sum\limits_{i=1}^m 1 = m$}
                \item \textcolor{gray}{$\sum\limits_{i=1}^m i = \frac{m(m+1)}{2}$}
                \item \textcolor{blue}{$\sum\limits_{i=a}^b c f(i) = c \sum\limits_{i=a}^b f(i)$ where $c$ is a \textbf{constant} and $f$ is a function}
                \item \textcolor{gray}{$\sum\limits_{i=a}^b (f(i) + g(i)) = \sum\limits_{i=a}^b f(i) + \sum\limits_{i=a}^b g(i)$ where $f$ and $g$ are functions}
            \end{enumerate}
        \end{column}
    \end{columns}
\end{frame}

% ------------------------------------------------------------
% Slide
% ------------------------------------------------------------

\begin{frame}
\frametitle{Simplifying the Cost Function}
    \begin{block}{Addition Cost Function}
        $T(n) = (n -1) + \textcolor{red}{\sum\limits_{i=1}^{n - 1} 2} + \sum\limits_{i=0}^{n-1} 1 + \sum\limits_{i=0}^{n - 1} \sum\limits_{j=0}^{i} 1 + \sum\limits_{i=0}^{n - 1} \sum\limits_{j=0}^{i - 1} 1$
    \end{block}

    \begin{columns}[T] 
        \begin{column}{0.4\textwidth}
            \textbf{Simplification:}

            Apply (3) with $a = 1$, $b = n - 1$, $c = 2$ and $f(i) = 1$
            \begin{align*}
                \sum\limits_{i=1}^{n - 1} 2 = \sum\limits_{i=1}^{n - 1} 2 \cdot 1 = 2\sum\limits_{i=1}^{n - 1} 1
            \end{align*}

            Apply \textcolor{blue}{(1)} with $m = n - 1$
            \begin{align*}
                \textcolor{red}{2\sum\limits_{i=1}^{n - 1} 1 = 2(n - 1)}
            \end{align*}
        \end{column}
    
        \begin{column}{0.6\textwidth}
            \textbf{Identities:}
            \begin{enumerate}
                \item \textcolor{blue}{$\sum\limits_{i=1}^m 1 = m$}
                \item \textcolor{gray}{$\sum\limits_{i=1}^m i = \frac{m(m+1)}{2}$}
                \item \textcolor{gray}{$\sum\limits_{i=a}^b c f(i) = c \sum\limits_{i=a}^b f(i)$ where $c$ is a \textbf{constant} and $f$ is a function}
                \item \textcolor{gray}{$\sum\limits_{i=a}^b (f(i) + g(i)) = \sum\limits_{i=a}^b f(i) + \sum\limits_{i=a}^b g(i)$ where $f$ and $g$ are functions}
            \end{enumerate}
        \end{column}
    \end{columns}
\end{frame}

% ------------------------------------------------------------
% Slide
% ------------------------------------------------------------

\begin{frame}
\frametitle{Simplifying the Cost Function}
    \begin{block}{Addition Cost Function}
        $T(n) = (n-1) + 2(n - 1) + \textcolor{red}{\sum\limits_{i=0}^{n-1} 1} + \sum\limits_{i=0}^{n - 1} \sum\limits_{j=0}^{i} 1 + \sum\limits_{i=0}^{n - 1} \sum\limits_{j=0}^{i - 1} 1$
    \end{block}

    \begin{columns}[T] 
        \begin{column}{0.4\textwidth}
            \textbf{Simplification:}
            \begin{align*}
                \textcolor{red}{\sum\limits_{i=0}^{n-1} 1 = \; ?}
            \end{align*}
            \begin{center}
                \Large
                \textcolor{red}{\textbf{What identity should we use?}}
            \end{center}
        \end{column}
    
        \begin{column}{0.6\textwidth}
            \textbf{Identities:}
            \begin{enumerate}
                \item \textcolor{black}{$\sum\limits_{i=1}^m 1 = m$}
                \item \textcolor{black}{$\sum\limits_{i=1}^m i = \frac{m(m+1)}{2}$}
                \item \textcolor{black}{$\sum\limits_{i=a}^b c f(i) = c \sum\limits_{i=a}^b f(i)$ where $c$ is a \textbf{constant} and $f$ is a function}
                \item \textcolor{black}{$\sum\limits_{i=a}^b (f(i) + g(i)) = \sum\limits_{i=a}^b f(i) + \sum\limits_{i=a}^b g(i)$ where $f$ and $g$ are functions}
            \end{enumerate}
        \end{column}
    \end{columns}
\end{frame}

% ------------------------------------------------------------
% Slide
% ------------------------------------------------------------

\begin{frame}
\frametitle{Simplifying the Cost Function}
    \begin{block}{Addition Cost Function}
        $T(n) = (n-1) + 2(n - 1) + \textcolor{red}{\sum\limits_{i=0}^{n-1} 1} + \sum\limits_{i=0}^{n - 1} \sum\limits_{j=0}^{i} 1 + \sum\limits_{i=0}^{n - 1} \sum\limits_{j=0}^{i - 1} 1$
    \end{block}

    \begin{columns}[T] 
        \begin{column}{0.4\textwidth}
            \textbf{Simplification:}

            Apply \textcolor{blue}{(1)} with $m = n-1$
            \begin{align*}
                \textcolor{red}{\sum\limits_{i=0}^{n-1} 1 = 1 + \sum\limits_{i=1}^{n-1} 1 = (n-1) + 1 = n}
            \end{align*}
        \end{column}
    
        \begin{column}{0.6\textwidth}
            \textbf{Identities:}
            \begin{enumerate}
                \item \textcolor{blue}{$\sum\limits_{i=1}^m 1 = m$}
                \item \textcolor{gray}{$\sum\limits_{i=1}^m i = \frac{m(m+1)}{2}$}
                \item \textcolor{gray}{$\sum\limits_{i=a}^b c f(i) = c \sum\limits_{i=a}^b f(i)$ where $c$ is a \textbf{constant} and $f$ is a function}
                \item \textcolor{gray}{$\sum\limits_{i=a}^b (f(i) + g(i)) = \sum\limits_{i=a}^b f(i) + \sum\limits_{i=a}^b g(i)$ where $f$ and $g$ are functions}
            \end{enumerate}
        \end{column}
    \end{columns}
\end{frame}

% ------------------------------------------------------------
% Slide
% ------------------------------------------------------------

\begin{frame}
\frametitle{Simplifying the Cost Function}
    \begin{block}{Addition Cost Function}
        $T(n) = (n-1) + 2(n - 1) + n + \textcolor{red}{\sum\limits_{i=0}^{n - 1} \sum\limits_{j=0}^{i} 1} + \sum\limits_{i=0}^{n - 1} \sum\limits_{j=0}^{i - 1} 1$
    \end{block}

    \begin{columns}[T] 
        \begin{column}{0.4\textwidth}
            \textbf{Simplification:}
            \begin{align*}
                \textcolor{red}{\sum\limits_{i=0}^{n - 1} \sum\limits_{j=0}^{i} 1 = \; ?}
            \end{align*}
            \begin{center}
                \Large
                \textcolor{red}{\textbf{What identity should we use?}}
            \end{center}
        \end{column}
    
        \begin{column}{0.6\textwidth}
            \textbf{Identities:}
            \begin{enumerate}
                \item \textcolor{black}{$\sum\limits_{i=1}^m 1 = m$}
                \item \textcolor{black}{$\sum\limits_{i=1}^m i = \frac{m(m+1)}{2}$}
                \item \textcolor{black}{$\sum\limits_{i=a}^b c f(i) = c \sum\limits_{i=a}^b f(i)$ where $c$ is a \textbf{constant} and $f$ is a function}
                \item \textcolor{black}{$\sum\limits_{i=a}^b (f(i) + g(i)) = \sum\limits_{i=a}^b f(i) + \sum\limits_{i=a}^b g(i)$ where $f$ and $g$ are functions}
            \end{enumerate}
        \end{column}
    \end{columns}
\end{frame}

% ------------------------------------------------------------
% Slide
% ------------------------------------------------------------

\begin{frame}
\frametitle{Simplifying the Cost Function}
    \begin{block}{Addition Cost Function}
        $T(n) = (n-1) + 2(n - 1) + n + \textcolor{red}{\sum\limits_{i=0}^{n - 1} \sum\limits_{j=0}^{i} 1} + \sum\limits_{i=0}^{n - 1} \sum\limits_{j=0}^{i - 1} 1$
    \end{block}

    \begin{columns}[T] 
        \begin{column}{0.4\textwidth}
            \textbf{Simplification:}

            Apply \textcolor{blue}{(1)} with $m = i$
            \begin{align*}
                \textcolor{red}{\sum\limits_{i=0}^{n - 1} \sum\limits_{j=0}^{i} 1} 
                &\textcolor{red}{= \sum\limits_{i=0}^{n - 1} \left( 1 + \sum\limits_{j=1}^{i} 1 \right)} \\
                &\textcolor{red}{= \sum\limits_{i=0}^{n - 1} \left( i + 1 \right)}
            \end{align*}
            \begin{center}
                \Large
                \textcolor{red}{\textbf{What identity should we use?}}
            \end{center}
        \end{column}
    
        \begin{column}{0.6\textwidth}
            \textbf{Identities:}
            \begin{enumerate}
                \item \textcolor{blue}{$\sum\limits_{i=1}^m 1 = m$}
                \item \textcolor{gray}{$\sum\limits_{i=1}^m i = \frac{m(m+1)}{2}$}
                \item \textcolor{gray}{$\sum\limits_{i=a}^b c f(i) = c \sum\limits_{i=a}^b f(i)$ where $c$ is a \textbf{constant} and $f$ is a function}
                \item \textcolor{gray}{$\sum\limits_{i=a}^b (f(i) + g(i)) = \sum\limits_{i=a}^b f(i) + \sum\limits_{i=a}^b g(i)$ where $f$ and $g$ are functions}
            \end{enumerate}
        \end{column}
    \end{columns}
\end{frame}

% ------------------------------------------------------------
% Slide
% ------------------------------------------------------------

\begin{frame}
\frametitle{Simplifying the Cost Function}
    \begin{block}{Addition Cost Function}
        $T(n) = (n-1) + 2(n - 1) + n + \textcolor{red}{\sum\limits_{i=0}^{n - 1} \sum\limits_{j=0}^{i} 1} + \sum\limits_{i=0}^{n - 1} \sum\limits_{j=0}^{i - 1} 1$
    \end{block}

    \begin{columns}[T] 
        \begin{column}{0.4\textwidth}
            \textbf{Simplification:}

            Apply \textcolor{blue}{(4)} with $a=0$, $b=n-1$, $f(i) = i$ and $g(i) = 1$
            \begin{align*}
                \textcolor{red}{\sum\limits_{i=0}^{n - 1} \left( i + 1 \right) = \sum\limits_{i=0}^{n - 1} i + \sum\limits_{i=0}^{n - 1} 1}
            \end{align*}
            \begin{center}
                \Large
                \textcolor{red}{\textbf{What identities should we use?}}
            \end{center}
        \end{column}
    
        \begin{column}{0.6\textwidth}
            \textbf{Identities:}
            \begin{enumerate}
                \item \textcolor{gray}{$\sum\limits_{i=1}^m 1 = m$}
                \item \textcolor{gray}{$\sum\limits_{i=1}^m i = \frac{m(m+1)}{2}$}
                \item \textcolor{gray}{$\sum\limits_{i=a}^b c f(i) = c \sum\limits_{i=a}^b f(i)$ where $c$ is a \textbf{constant} and $f$ is a function}
                \item \textcolor{blue}{$\sum\limits_{i=a}^b (f(i) + g(i)) = \sum\limits_{i=a}^b f(i) + \sum\limits_{i=a}^b g(i)$ where $f$ and $g$ are functions}
            \end{enumerate}
        \end{column}
    \end{columns}
\end{frame}

% ------------------------------------------------------------
% Slide
% ------------------------------------------------------------

\begin{frame}
\frametitle{Simplifying the Cost Function}
    \begin{block}{Addition Cost Function}
        $T(n) = (n - 1) + 2(n - 1) + n + \textcolor{red}{\sum\limits_{i=0}^{n - 1} \sum\limits_{j=0}^{i} 1} + \sum\limits_{i=0}^{n - 1} \sum\limits_{j=0}^{i - 1} 1$
    \end{block}

    \begin{columns}[T] 
        \begin{column}{0.4\textwidth}
            \textbf{Simplification:}

            Apply \textcolor{blue}{(1)} with $m = n - 1$
            \begin{align*}
                \textcolor{red}{\sum\limits_{i=0}^{n - 1} 1} 
                &\textcolor{red}{= 1 + \sum\limits_{i=1}^{n - 1} 1} \\
                &\textcolor{red}{= 1 + (n - 1) = n}
            \end{align*}
            so
            \begin{align*}
                \textcolor{red}{\sum\limits_{i=0}^{n - 1} i + \sum\limits_{i=0}^{n - 1} 1 = n + \sum\limits_{i=0}^{n - 1} i}
            \end{align*}
            
        \end{column}
    
        \begin{column}{0.6\textwidth}
            \textbf{Identities:}
            \begin{enumerate}
                \item \textcolor{blue}{$\sum\limits_{i=1}^m 1 = m$}
                \item \textcolor{gray}{$\sum\limits_{i=1}^m i = \frac{m(m+1)}{2}$}
                \item \textcolor{gray}{$\sum\limits_{i=a}^b c f(i) = c \sum\limits_{i=a}^b f(i)$ where $c$ is a \textbf{constant} and $f$ is a function}
                \item \textcolor{gray}{$\sum\limits_{i=a}^b (f(i) + g(i)) = \sum\limits_{i=a}^b f(i) + \sum\limits_{i=a}^b g(i)$ where $f$ and $g$ are functions}
            \end{enumerate}
        \end{column}
    \end{columns}
\end{frame}

% ------------------------------------------------------------
% Slide
% ------------------------------------------------------------

\begin{frame}
\frametitle{Simplifying the Cost Function}
    \begin{block}{Addition Cost Function}
        $T(n) = (n -1) + 2(n - 1) + n + \textcolor{red}{\sum\limits_{i=0}^{n - 1} \sum\limits_{j=0}^{i} 1} + \sum\limits_{i=0}^{n - 1} \sum\limits_{j=0}^{i - 1} 1$
    \end{block}

    \begin{columns}[T] 
        \begin{column}{0.4\textwidth}
            \textbf{Simplification:}

            Apply \textcolor{blue}{(2)} with $m = n - 1$
            \begin{align*}
                \textcolor{red}{\sum\limits_{i=0}^{n - 1} i} 
                &\textcolor{red}{= 0 + \sum\limits_{i=1}^{n - 1} i} \\
                &\textcolor{red}{= \frac{(n - 1)((n-1) + 1)}{2}}
            \end{align*}
            so
            \begin{align*}
                \textcolor{red}{n + \sum\limits_{i=0}^{n - 1} i = n + \frac{n(n - 1)}{2}}
            \end{align*}
            
        \end{column}
    
        \begin{column}{0.6\textwidth}
            \textbf{Identities:}
            \begin{enumerate}
                \item \textcolor{gray}{$\sum\limits_{i=1}^m 1 = m$}
                \item \textcolor{blue}{$\sum\limits_{i=1}^m i = \frac{m(m+1)}{2}$}
                \item \textcolor{gray}{$\sum\limits_{i=a}^b c f(i) = c \sum\limits_{i=a}^b f(i)$ where $c$ is a \textbf{constant} and $f$ is a function}
                \item \textcolor{gray}{$\sum\limits_{i=a}^b (f(i) + g(i)) = \sum\limits_{i=a}^b f(i) + \sum\limits_{i=a}^b g(i)$ where $f$ and $g$ are functions}
            \end{enumerate}
        \end{column}
    \end{columns}
\end{frame}

% ------------------------------------------------------------
% Slide
% ------------------------------------------------------------

\begin{frame}
\frametitle{Simplifying the Cost Function}
    \begin{block}{Addition Cost Function}
        $T(n) = (n-1) + 2(n - 1) + n + \left(n + \frac{n(n - 1)}{2}\right) + \textcolor{red}{\sum\limits_{i=0}^{n - 1} \sum\limits_{j=0}^{i - 1} 1}$
    \end{block}

    \begin{columns}[T] 
        \begin{column}{0.4\textwidth}
            \textbf{Simplification:}

            \begin{align*}
                \textcolor{red}{\sum\limits_{i=0}^{n - 1} \sum\limits_{j=0}^{i - 1} 1 = \; ?}
            \end{align*}
            \begin{center}
                \Large
                \textcolor{red}{\textbf{What is the simplified form?}}
            \end{center}
            
        \end{column}
    
        \begin{column}{0.6\textwidth}
            \textbf{Identities:}
            \begin{enumerate}
                \item \textcolor{black}{$\sum\limits_{i=1}^m 1 = m$}
                \item \textcolor{black}{$\sum\limits_{i=1}^m i = \frac{m(m+1)}{2}$}
                \item \textcolor{black}{$\sum\limits_{i=a}^b c f(i) = c \sum\limits_{i=a}^b f(i)$ where $c$ is a \textbf{constant} and $f$ is a function}
                \item \textcolor{black}{$\sum\limits_{i=a}^b (f(i) + g(i)) = \sum\limits_{i=a}^b f(i) + \sum\limits_{i=a}^b g(i)$ where $f$ and $g$ are functions}
            \end{enumerate}
        \end{column}
    \end{columns}
\end{frame}

% ------------------------------------------------------------
% Slide
% ------------------------------------------------------------

\begin{frame}
\frametitle{Simplifying the Cost Function}
    \begin{block}{Addition Cost Function}
        $T(n) = (n-1) + 2(n - 1) + n + \left(n + \frac{n(n - 1)}{2}\right) + \textcolor{red}{\sum\limits_{i=0}^{n - 1} \sum\limits_{j=0}^{i - 1} 1}$
    \end{block}

    \begin{columns}[T] 
        \begin{column}{0.4\textwidth}
            \textbf{Simplification:}

            Apply \textcolor{blue}{(1)} then \textcolor{blue}{(2)}            
            \begin{align*}
                \textcolor{red}{\sum\limits_{i=0}^{n - 1} \sum\limits_{j=0}^{i - 1} 1 = \sum\limits_{i=0}^{n - 1} i = \frac{n(n - 1)}{2}}
            \end{align*}

        \end{column}
    
        \begin{column}{0.6\textwidth}
            \textbf{Identities:}
            \begin{enumerate}
                \item \textcolor{blue}{$\sum\limits_{i=1}^m 1 = m$}
                \item \textcolor{blue}{$\sum\limits_{i=1}^m i = \frac{m(m+1)}{2}$}
                \item \textcolor{gray}{$\sum\limits_{i=a}^b c f(i) = c \sum\limits_{i=a}^b f(i)$ where $c$ is a \textbf{constant} and $f$ is a function}
                \item \textcolor{gray}{$\sum\limits_{i=a}^b (f(i) + g(i)) = \sum\limits_{i=a}^b f(i) + \sum\limits_{i=a}^b g(i)$ where $f$ and $g$ are functions}
            \end{enumerate}
        \end{column}
    \end{columns}
\end{frame}

% ------------------------------------------------------------
% Slide
% ------------------------------------------------------------

\begin{frame}
\frametitle{Simplifying the Cost Function}

    \textbf{Simplification:}
    
    We have that
    \begin{align*}
        T(n) &= (n-1) + 2(n - 1) + n + \left(n + \frac{n(n - 1)}{2}\right) + \frac{n(n-1)}{2} \\
        &= n - 1 + 2n - 2 + n + n + n(n-1) \\
        &= n^2 + 4n - 3  \\
    \end{align*}
    so the final answer is
    \begin{center}
        \textbf{\textcolor{blue}{$T(n) = n^2 + 4n - 3$}}
    \end{center}
\end{frame}

% ------------------------------------------------------------
% Slide
% ------------------------------------------------------------

\begin{frame}
\frametitle{Simplifying the Cost Function}
    \begin{block}{Multiplication Cost Function}
        $T(n) = \sum_{i=1}^{n - 1} 1$
    \end{block}

    \begin{block}{Comparisons Cost Function}
        $T(n) = \sum_{i=1}^{n} 1 + \sum_{i=0}^{n} 1 + \sum_{i=0}^{n-1} \sum_{j=0}^{i} 1$
    \end{block}

    \begin{block}{Memory Accesses Cost Function}
        $T(n) = 1 + \sum_{i=1}^{n-1} 2 + \sum_{i=0}^{n-1} \sum_{j=0}^{i - 1} 1$
    \end{block}

    \begin{block}{Assignments Cost Function}
        $T(n) = 3 + \sum_{i=1}^{n-1} 1 + \sum_{i=1}^{n-1} 1 + \sum_{i=0}^{n-1} 1 + \sum_{i=0}^{n-1} \sum_{j=0}^{i} 1 + \sum_{i=0}^{n-1} \sum_{j=0}^{i - 1} 1$
    \end{block}

    \begin{center}
        \Large
        \textcolor{red}{\textbf{Try to simplify the other cost functions!}}
    \end{center}
\end{frame}

% ------------------------------------------------------------
% Slide
% ------------------------------------------------------------

\begin{frame}
\frametitle{Simplifying the Cost Function}
    \begin{block}{Multiplication Cost Function}
        \begin{align*}
            \textcolor{blue}{T(n)} 
            &\textcolor{blue}{= \sum_{i=1}^{n - 1} 1} \\
            &\textcolor{blue}{= n - 1}
        \end{align*}
    \end{block}
\end{frame}


% ------------------------------------------------------------
% Slide
% ------------------------------------------------------------

\begin{frame}
\frametitle{Simplifying the Cost Function}
    \begin{block}{Comparisons Cost Function}
        \begin{align*}
            \textcolor{blue}{T(n)} 
            &= \textcolor{blue}{\sum_{i=1}^{n} 1 + \sum_{i=0}^{n} 1 + \sum_{i=0}^{n-1} \sum_{j=0}^{i} 1} \\
            &= \textcolor{blue}{n + (n + 1) + \frac{n(n+1)}{2}} \\
            &= \textcolor{blue}{\frac{1}{2} n^2 + \frac{5}{2} n + 1}
        \end{align*}
    \end{block}
\end{frame}

% ------------------------------------------------------------
% Slide
% ------------------------------------------------------------

\begin{frame}
\frametitle{Simplifying the Cost Function}
    \begin{block}{Memory Accesses Cost Function}
        \begin{align*}
            \textcolor{blue}{T(n)} 
            &= \textcolor{blue}{1 + \sum_{i=1}^{n-1} 2 + \sum_{i=0}^{n-1} \sum_{j=0}^{i - 1} 1} \\
            &= \textcolor{blue}{1 + 2(n-1) + \frac{n(n-1)}{2}} \\
            &= \textcolor{blue}{\frac{1}{2} n^2 + \frac{3}{2} n -1}
        \end{align*}
    \end{block}
\end{frame}


% ------------------------------------------------------------
% Slide
% ------------------------------------------------------------

\begin{frame}
\frametitle{Simplifying the Cost Function}
    \begin{block}{Assignments Cost Function}
        \begin{align*}
            \textcolor{blue}{T(n)} 
            &= \textcolor{blue}{3 + \sum_{i=1}^{n-1} 1 + \sum_{i=1}^{n-1} 1 + \sum_{i=0}^{n-1} 1 + \sum_{i=0}^{n-1} \sum_{j=0}^{i} 1 + \sum_{i=0}^{n-1} \sum_{j=0}^{i - 1} 1} \\
            &= \textcolor{blue}{3 + (n-1) + (n - 1) + n + \frac{n(n+1)}{2} + \frac{n(n-1)}{2}} \\
            &= \textcolor{blue}{n^2 + 3n + 1}
        \end{align*}
    \end{block}
\end{frame}

% ------------------------------------------------------------
% Slide
% ------------------------------------------------------------

\begin{frame}[fragile]
\frametitle{Simplifying the Cost Function}
   \begin{center}
        \Large
        \textcolor{red}{\textbf{Pitfall: Cannot represent loops with non-one increments using standard summations!}}
    \end{center}

    \begin{lstlisting}[language=C++]
int baz(int n) {
    int total = 0;
    for (int i = 0; i < n; i += 2)
        total += i * i;
    return total;
}
    \end{lstlisting}

    
    \begin{center}
        \Large
        \textcolor{blue}{\textbf{Solution: Compute number of operations for small values of $n$, then guess and check the formula OR learn Knuth's summation notation (advanced)!}}
    \end{center}
\end{frame}

% ------------------------------------------------------------
% Slide
% ------------------------------------------------------------

\begin{frame}[fragile]
\frametitle{Extra Practice}
    \begin{columns}[T] 
    
        \begin{column}{0.65\textwidth}
            \begin{lstlisting}[language=C++]
int bar(int n) {
    int* A = new int[n];

    for (int i = 0; i < n; i++)
        A[i] = i + 1;

    int result = 0;
    for (int k = 0; k < n * n; k++)
        for (int j = 0; j <= k; j++)
            for (int i = j; i < n; i++)
                result += A[i];

    for (int t = 0; t < 7; t++)
        result += A[0];

    delete[] A;
    return result;
}
           \end{lstlisting}
        \end{column}
    
        \begin{column}{0.35\textwidth}
            \begin{center}
                 \textcolor{red}{\textbf{Choose a model of computation and analyze this algorithm!}}
            \end{center}
        \end{column}
    \end{columns}
\end{frame}

\end{document}